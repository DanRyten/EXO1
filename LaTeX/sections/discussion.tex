\section{Discussion}
\label{section:disc}

Looking back at the research questions,

\textbf{RQ1:} Can the elbow actuator be given three distinct states of function, turning the arm upwards, downwards, and keeping it at rest?

\textbf{RQ2:} Can the elbow actuator operate at variable speeds based on the effort exerted by the wearer?

\textbf{RQ3:} Can the latency between the action of the wearer and the action of the actuator be reduced?

The structure for the model was constructed to facilitate RQ1 but the training was not functional.

For RQ2 the method for driving the actuators at variable speeds was not implemented but could be feasibly implemented as a subVI in Labview. 
A possible to implement this would be by analyzing the amplitude of the inputs or comparing the outputs of the RNN, with higher certainty leading to higher speed 

Lastly, for RQ3 a full iteration of the RNN in LabView took 230 ms, showing a clear improvement compared to the testing of the exoskeleton 
made by the last group, the exoskeleton finished a full movement from start to stop approximately one second after the user finished the 
movement based on observation of a video from the previous group.
