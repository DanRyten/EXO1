\section{Results}
\label{section:results}

%This section should present answers to all research questions.
%
%It is normal to have only one results section, but you can create more sections if finding it more appropriate. You can
%also divide results into subsections. Perhaps you want to refer to some other section, for example (see Section \ref{section:method}).
%You can also place figures, you should always reference these in the text, see Figure \ref{fig:MDHlogga} for an example of a figure
%including subfigures. Remember that all figures should have a figure label explaining their content.
%
%
%\begin{figure}[H]
%    \centering
%    \subfigure[MDH logo]{
%    \label{fig:MDH1}
%    \includegraphics[width=.2\columnwidth]{MDHlogga}}
%    \qquad
%    \subfigure[MDH logo]{
%    \label{fig:MDH2}
%    \includegraphics[width=.2\columnwidth]{MDHlogga}}
%    \qquad
%    \subfigure[MDH logo]{
%    \label{fig:MDH3}
%    \includegraphics[width=.2\columnwidth]{MDHlogga}}
%    \caption[Short text]{This is an example of multiple figures.}
%    \label{fig:MDHlogga}
%\end{figure}
Looking back at the research questions,

\textbf{RQ1:} Can the elbow actuator be given three distinct states of function, turning the arm upwards, downwards, and keeping it at rest?

\textbf{RQ2:} Can the elbow actuator operate at variable speeds based on the effort exerted by the wearer?

\textbf{RQ3:} Can the latency between the action of the wearer and the action of the actuator be reduced?

The results showed that for RQ1 the \acrshort{rnn} in Python was not able to output the three distinct states of 
function for the arm since both \acrshort{rnn} models achieved a precision of 25\% on the validation set no matter the
learning rate or number of epochs used. Additionally, the results consisted of the model only giving one single class as the output
and therefore having a 100\% accuracy on that specific class and a 0\% accuracy on the rest of the classes. This also resulted in 
the correct weights not being transferred to Labview.

For RQ2 the method for driving the actuators at variable speeds was not implemented but could be easily implemented as a subVI in Labview.  

Lastly, for RQ3 the time it takes to run one iteration of the code in LabView was 230ms.