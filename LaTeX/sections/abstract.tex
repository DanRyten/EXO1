\begin{abstract}
This project deals with creating a controll system for an exoskeleton using \acrshort{semg} processed by a \acrshort{rnn} in LabView
allowing for assistance during static work.
Though the training of the \acrshort{rnn} was incomplete, the speed of it shows definite improvement and the necessary structure
for a functioning controll unit is precent.




Physically demanding work often results in strain-related health issues, impairing individuals' ability to perform daily tasks and potentially causing long-term disability due to their long working hours. 
Exoskeletons offer a solution that can relieve the workers from some of the load during the workdays, but exoskeletons do not come cheap their high cost and person-specific designs limit their availability. 
This project aims to develop a commercially viable powered exoskeleton to assist with arm movements, specifically focusing on elbow actuation. 
The research addresses three primary questions: achieving three distinct elbow actuator states (upward, downward, rest), enabling variable speed function depending on the user effort, and reducing response latency between the user's actions and the actuator. 
The study involved using sEMG sensors, the RNN AI method. While achieving partial success, the project identified key areas for future improvement, including better data handling and system optimization for real-time applications.
\end{abstract}