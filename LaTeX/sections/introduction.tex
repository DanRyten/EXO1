\section{Introduction}
\label{section:intro}

Physically demanding work can leave people with such severe strain that they have to work less or stop working entirely in 
their field\cite{WorkDemands}. The burdens they are placed under can be alleviated using exoskeletons but they are often 
prohibitively expensive both from a material cost and since most are person-specific. The majority of commercially available 
exoskeletons are passive providing no extra power of their own but simply redistributing the load instead. The goal of this 
project is to advance the possibility of commercially viable exoskeletons that can assist workers in these fields by aiding 
them in bending and extending their arms as well as holding them in place.

%What are exoskeletons? Probably more background than introduction
Despite the word originating from nature, in robotics, the field researched by this project, an exoskeleton is an active wearable 
mechanical device that augments the user's performance \cite{ExoDefinition}. These can be divided into two categories, active and passive. 
While passive exoskeletons do not make use of external power sources and use materials such as springs or dampers, active ones employ 
one or more actuators (e.g., motors) to provide assistance to the user \cite{PassiveActiveExo}.

%What is EMG? Same as before, looks more like background
The movement of the human body involves numerous electric signals. These electric currents called \ac{emg} %[NOTE: EMG is the method to sense the signals not the signals themselves] 
are generated by the brain in order to contract the muscles and travel through motor neurons until they reach the motor unit, which is composed of the muscle fibers
activated by a single motor neuron \cite{EMGgen}. In order to measure these signals, two types of electrodes can be used, \ac{semg}, 
which consists of electrodes that should be placed on the skin above the muscle of interest.
needle electrodes which are placed within the muscle. 
%TODO: Add more about EMG and its usefulness and relevance to the project

Introduction:
    Introduce the reader:
        What are exoskeletons?
        What is EMG?
    Motivation:
        Static struggles!
            Medical work:
                Assistance while having to hold patients in exact positions for long durations during surgeries

            Industrial work:
                Assistance while working in awkward positions like overhead or under the hood of a car for example

            Agricultural work:
                Assistance while working overhead to pick for example mangos
            
            All above:
                Helps while carrying stuff

    Describe the subject:
        Idk, stuff?

    Aim of the thesis work:
        Problem formulation: %This sounds a lot like aim of the thesis work so I added it here
            Research questions:
                1.	Can the exoskeleton be driven based off of both the bicep and triceps? (So that it can be put at rest at a specified position)

                2.	Can the exoskeleton be made to drive at variable speeds?

                3.	Can the exoskeleton be made wearable?

            Problem:
                

            Scope:
            
            Relevance: %this is mostly motivation I guess


Background: 
    Short background on the research field:

    Related work: %Probably more could be written about what it achied. Rn it feels harsher than it should be
        This project is based on the work of \cite{AFES}, which achieved a working exoskeleton that could be driven by the bicep. However, the resulting 
        exoskeleton presented some limitations. The exoskeleton was mounted on a stand, rendering it therefore unwearable. Another significant limitation 
        that the exoskeleton exhibits lies in the fact that it can only be driven by the bicep. As such, the exoskeleton can only show three different states, 
        with them being fully extended, fully contracted, and at rest. One key issue that made the exoskeleton a valuable proof of concept, but not a viable 
        product consists of the limitations of the hardware and software used. The EMG sensor used employed an outdated Bluetooth module and the software was 
        written in Python, which due to its performance is not suitable for real-time applications in such limited hardware.

    Research platforms/groups:

    State of the art:


    This will present an overview of the project and what is done in a short intro


The goals of this project are threefold:

1. Can the elbow actuator be given three distinct states of function, turning the arm upwards, downwards, and keeping it at rest?

2. Can the elbow actuator drive at variable speeds related to the effort expended by the wearer?

3. Can the latency of the actuator acting be reduced?

This is to be accomplished through the use of additional sensors, updated hardware, as well as a change of programming language.