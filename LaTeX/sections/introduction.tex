\section{Introduction}
\label{section:intro}

When performing physically demanding work, people can be left with strain-based issues. These issues can range from mild issues to more severe issues which can impair their ability to function. 
This can result in permanent loss of function or not being able to perform the required work tasks for their respective field \cite{WorkDemands}. To prevent further harm, and alleviate the burden they are placed under, exoskeletons can be used.
The main drawback of these exoskeletons is that they are expensive, both from a material cost and since most are person-specific. The majority of commercially available 
exoskeletons are passive, providing no extra power of their own but simply redistributing the load. The goal of this 
project is to advance the possibility of commercially viable powered exoskeletons that can assist workers in these fields by aiding 
them in bending and extending their arms as well as holding them in place continuing the work of \acrfull{afes} \cite{AFES}. They achieved a working prototype with two degrees of freedom of which only one was hooked up. 
This was done with one \acrshort{emg} sensor and an SVM implemented in Python, running on a PC.

\subsection{Problem Formulation}
The goal of this research is to improve exoskeletons to better assist the user in bending, extending and positioning of the arm.
To look into this goal the following sub-research questions were posed:
\begin{itemize}
        
    \item \textbf{RQ1:} Can the elbow actuator be given three distinct states of function, turning the arm upwards, downwards, and keeping it at rest?

    \item \textbf{RQ2:} Can the elbow actuator operate at variable speeds based on the effort exerted by the wearer?

    \item \textbf{RQ3:} Can the latency between the action of the wearer and the action of the actuator be reduced?

\end{itemize}
RQ1 is the item of most importance since that would allow workers to receive support for static tasks instead of only receiving dynamic
assistance allowing for the user to functionally rest while at work. RQ2 would allow the user to start and stop the
movement of the exoskeleton gradually rather than discreetly which would allow the user to find the exact position they
wish to hold more easily, while RQ3 would allow for the exoskeleton to be used in real-time. These questions would be
accomplished by using additional sensors, a different choice of AI method as well as updated hardware and a different
choice of programming language.\newline


\subsection{Limitations}
This project was subject to several constraints. The overarching one was the limited time of the project, this being 8 weeks,
which was exasperated by the need to procure sensors, DAQ systems and applicable software. This factor mainly delayed
the data acquisition. Although the exoskeleton was assembled with two degrees of freedom only the elbow joint was used.
This was due to the larger complexity of shoulder movement requiring additional sensors and a larger system on all
fronts.
