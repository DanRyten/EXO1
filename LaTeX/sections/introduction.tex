\section{Introduction}
\label{section:intro}

Physically demanding work can leave people with such severe strain that they have to work less or stop working entirely in 
their field\cite{WorkDemands}. The burdens they are placed under can be alleviated using exoskeletons but they are often 
prohibitively expensive both from a material cost and since most are person-specific. The majority of commercially available 
exoskeletons are passive providing no extra power of their own but simply redistributing the load instead. The goal of this 
project is to advance the possibility of commercially viable exoskeletons that can assist workers in these fields by aiding 
them in bending and extending their arms as well as holding them in place.

\subsection{Problem Formulation}
The goals of this project are here listed in descending order of priority:
\begin{itemize}
        
    \item RQ1: Can the elbow actuator be given three distinct states of function, turning the arm upwards, downwards, and keeping it at rest?

    \item RQ2: Can the elbow actuator operate at variable speeds based on the effort exerted by the wearer?

    \item RQ3: Can the latency between the action of the wearer and the action of the actuator be reduced?

\end{itemize}
RQ1 is the item of most import since that would allow workers to accomplish static tasks instead of only getting dynamic
assistance allowing for the user to functionally rest while at work. RQ2 would allow the user to start and stop the
movement of the exoskeleton gradually rather than discreetly which would allow the user to find the exact position they
wish to hold more easily, while RQ3 would allow for the exoskeleton to be used in real-time. These questions would be
accomplished by using additional sensors, a different choice of AI method as well as updated hardware and a different
choice of programming language.



\subsubsection{Limitations}
This project was subject to several constraints from the beginning. The overarching one was the limited time of 8 weeks
which was exasperated by the need to procure sensors, DAQ systems and applicable software. This factor mainly delayed
the data acquisition. Although the exoskeleton was assembled with two degrees of freedom only the elbow joint was used.
This was due to the larger complexity of shoulder movement requiring additional sensors and a larger system on all
fronts.
