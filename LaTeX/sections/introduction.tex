\section{Introduction}
\label{section:intro}

Physically demanding work can leave people with such severe strain that they have to work less or stop working entirely in 
their field\cite{WorkDemands}. The burdens they are placed under can be alleviated using exoskeletons but they are often 
prohibitively expensive both from a material cost and since most are person-specific. The majority of commercially available 
exoskeletons are passive providing no extra power of their own but simply redistributing the load instead. The goal of this 
project is to advance the possibility of commercially viable exoskeletons that can assist workers in these fields by aiding 
them in bending and extending their arms as well as holding them in place.

\subsection{Problem Formulation and Limitations}
The goals of this project are here listed in descending order of priority:
\begin{itemize}
        
    \item RQ1: Can the elbow actuator be given three distinct states of function, turning the arm upwards, downwards, and keeping it at rest?

    \item RQ2: Can the elbow actuator drive at variable speeds related to the effort expended by the wearer?

    \item RQ3: Can the latency of the actuator acting be reduced?

\end{itemize}
RQ1 is the item of most import since that would allow workers to accomplish static tasks instead of only getting dynamic assistance 
allowing for the user to functionally rest while at work. RQ2 would allow for the user to start and stop more carefully so that they 
find the exaxct position they wish to hold more easily, while RQ3 would allow for the exoskeleton to be used in realtime.
This would be accoplished through additional sensors, a different choice of AI method as well as updated hardware and a different 
choice of programming language.




This project was subject to several restraints from the beginning. The overarching one was the limited time of 8 weeks which was
exasperated by the need to procure sensors, DAQ systems and applicable software. This factor mainly delayed the data acquisition. 
Although the exoskeleton was assembled with degrees of freedom only the elbow joint was used. This was due to the larger complexity 
of a shoulder movement requiring additional sensors and a larger system on all fronts.


Background: 
    Short background on the research field:

    Related work: %Probably more could be written about what it achied. Rn it feels harsher than it should be
        This project is based on the work of \cite{AFES}, which achieved a working exoskeleton that could be driven by the bicep. However, the resulting 
        exoskeleton presented some limitations. The exoskeleton was mounted on a stand, rendering it therefore unwearable. Another significant limitation 
        that the exoskeleton exhibits lies in the fact that it can only be driven by the bicep. As such, the exoskeleton can only show three different states, 
        with them being fully extended, fully contracted, and at rest. One key issue that made the exoskeleton a valuable proof of concept, but not a viable 
        product consists of the limitations of the hardware and software used. The \acs{emg} sensor used employed an outdated Bluetooth module and the software was 
        written in Python, which due to its performance is not suitable for real-time applications in such limited hardware. 
        \\\\
        One study "Real-time Multiple-Channel Shoulder EMG Processing for a Rehabilitative Upper-limb Exoskeleton Motion Control Using ANN Machine Learning" \cite{shoulderexo},
        worked on a 12-channel shoulder sEMG system with the use of machine learning techniques. Through the testing they used Delsys EMG sensors, LABVIEW with ML toolbox and an 
        upper limb exoskeleton where a DAQ and arduino boards was used for motion control. For the recognition of the shoulder motion and the different patterns ANN was used as an 
        ML technique. During testing the accuracy of the movement sensors was controlled with the used of eighteen test subjects.
    Research platforms/groups:

    State of the art:


    This will present an overview of the project and what is done in a short intro
