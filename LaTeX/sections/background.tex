\section{Background}
\label{section:background}

\subsection{Hardware}

%What are exoskeletons? Probably more background than introduction
Despite the word originating from nature, in robotics, the field researched by this project, an exoskeleton is an active wearable 
mechanical device that augments the user's performance \cite{exodefinition}. These can be divided into two categories, active and passive. 
While passive exoskeletons do not make use of external power sources and use materials such as springs or dampers, active ones employ 
one or more actuators (e.g., motors) to provide assistance to the user \cite{passiveactiveexo}.


\subsection{Software related concepts}

\subsubsection{\acf{emg}}

%What is EMG? Same as before, looks more like background
The movement of the human body involves numerous electric signals. These electric currents called \ac{emg} are generated by the brain
in order to contract the muscles and travel through motor neurons until they reach the motor unit, which is composed of the muscle fibers
activated by a single motor neuron \cite{emggen}. In order to measure these signals, two types of electrodes can be used, \ac{semg}, 
which consists on electrodes that should be placed on the skin above the muscle of interest or needle electrodes which are placed within 
the muscle. 
%TODO: Add more about EMG and its usefulness and relevance to the project

\subsubsection{\acf{semg}}

Surface electrodes are frequently made out of silver/silver chloride (Ag/AgCl), silver chloride (AgCl), silver (Ag) or gold (Au) \cite{sEMG}.
Silver/Silver chloride electrodes are more preferable than the others since they are almost non-polarizable, meaning that instead of capacitive
they are resistive in regards to the impedance between the skin and the electrode. This provides a highly stable interface with the skin when 
an electrolyte solution is used. Such stability is crucial for the quality of the signal as it lowers the noise and reduces the body movent generated
artifacts.
TEXT

\subsubsection{\acs{emg} amplitude and frequency}

TEXT

%¿\subsubsection{AI}?