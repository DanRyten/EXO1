\section{Background}
\label{section:background}

\subsection{Hardware}

Despite the word originating from nature, in robotics, the field researched by this project, an exoskeleton is an active wearable 
mechanical device that augments the user's performance \cite{exodefinition}. These can be divided into two categories, active and passive. 
While passive exoskeletons do not make use of external power sources and use materials such as springs or dampers, active ones employ 
one or more actuators (e.g., motors) to provide assistance to the user \cite{passiveactiveexo}.


\subsection{Software related concepts}

\subsubsection{\acrfull{emg}}

The movement of the human body involves numerous electric signals. These electric currents are generated by the brain
in order to contract the muscles and travel through motor neurons until they reach the motor unit, which is composed of the muscle fibers
activated by a single motor neuron \cite{emggen}. In order to measure these signals, two types of electrodes can be used, \acrshort{semg}, 
which consists on electrodes that should be placed on the skin above the muscle of interest or needle electrodes which are placed within 
the muscle. 

\subsubsection{\acrfull{semg}}

Surface electrodes are frequently made out of silver/silver chloride (Ag/AgCl), silver chloride (AgCl), silver (Ag) or gold (Au) \cite{sEMG}.
Silver/Silver chloride electrodes are more preferable than the others since they are almost non-polarizable, meaning that instead of capacitive
they are resistive in regards to the impedance between the skin and the electrode. This provides a highly stable interface with the skin when 
an electrolyte solution is used. Such stability is crucial for the quality of the signal as it lowers the noise and reduces the body movent generated
artifacts.

\subsubsection{\acrshort{emg} feature extraction}

TEXT

\subsubsection{Continuous motion prediction}

EMG-based continuous motion prediction can be categorized into two main approaches: model-based and model-free \cite{EMGprediction}. Model-based approaches
establish a linear or non-linear relationship between the EMG signals and the motion of interest. On the other hand, model-free approaches
employ a numerical function, commonly approximated by \acrshort{ml}, that maps the EMG inputs to the desired motion outputs. \cite{ANNFuzzy} shows how most 
researchers are now focusing on \acrshort{ml} techniques such as \acrshort{knn}, \acrshort{lda}, \acrshort{ann} and \acrshort{svm} to predict continuous motion from EMG signals.


\subsection{Related Works}
Related work: %Probably more could be written about what it achied. Rn it feels harsher than it should be
This project is based on the work of \cite{AFES}, which achieved a working exoskeleton that could be driven by the bicep. However, the resulting 
exoskeleton presented some limitations. The exoskeleton was mounted on a stand, rendering it therefore unwearable. Another significant limitation 
that the exoskeleton exhibits lies in the fact that it can only be driven by the bicep. As such, the exoskeleton can only show three different states, 
with them being fully extended, fully contracted, and at rest. One key issue that made the exoskeleton a valuable proof of concept, but not a viable 
product consists of the limitations of the hardware and software used. The \acrshort{emg} sensor used employed an outdated Bluetooth module and the software was 
written in Python, which due to its performance is not suitable for real-time applications in such limited hardware. 
\\\\
One study "Real-time Multiple-Channel Shoulder EMG Processing for a Rehabilitative Upper-limb Exoskeleton Motion Control Using ANN Machine Learning" \cite{shoulderexo},
worked on a 12-channel shoulder sEMG system with the use of machine learning techniques. Through the testing they used Delsys EMG sensors, LABVIEW with ML toolbox and an 
upper limb exoskeleton where a DAQ and arduino boards was used for motion control. For the recognition of the shoulder motion and the different patterns ANN was used as an 
ML technique. During testing the accuracy of the movement sensors was controlled with the used of eighteen test subjects.