\section{Conclusion}


Due to unnoticed bugs the training that had been done was found to be invalid and could not be rectified in time.
The
weights of the RNN could not be transferred to LabView without great difficulties due to the lack of clarity provided by
pyTorch. Making use of methods that would allow for either easier weight extraction or for the model to be trained
directly in LabView would be preferred.


Due to time constraints the method for driving them at variable speeds was not implemented. The steps to implement it
would involve either a separate function analyzing the amplitude of the inputs, or to compare the outputs of the RNN,
with higher certainty leading to higher speed.


A definite improvement from the starting point of the previous work, showing an improvement by ???\%. (MIGHT WANT TO TRY TO MEASURE THE TIME A BIT MORE
PROPERLY FROM THE VIDEO)


