\section{Conclusion}
Observing the results given we can conclude that with the use of the Myoware sensors and Analog Discovery studio, the signal Acquisition was processed without large issues since the signals received from the sensors were consistent from user to user. 
The Analog Discovery studio studio made processing and transferring the signals to Python lightwork.


An unprocessed signal was fed into the \acrshort{rnn} due to the no significant features found in the feature extraction which seemed worth the processing time.


After inspecting both model implementations and the data used to train and test them, it was concluded that the problem behind the accuracy
of the models might be the data itself. The data files were shuffled before being fed into the model to avoid learning on the test subject
itself, however, the data was not preprocessed in any other way aside from windowing. The data from each file was fed into the model in the 
same order as it was collected, which followed a consistent class order. This consistent class order might have led the model's gradient to 
get stuck in a local minimum, which would explain the 25\% accuracy on the validation set.


Another aspect to take into account in regards to the training of the \acrshort{rnn} is how the data is fed to the model. The issues
with the training could be caused by the repetitive nature of the data, where the same movement order is repeated. Splitting the data
into atomic movements and then combining them in a random order could be a solution to this issue.


The weights of the \acrshort{rnn} could not be transferred to LabView without great difficulties due to the lack of clarity provided by
PyTorch. Making use of methods that would allow for either easier weight extraction or for the model to be trained
directly in LabView would be preferred.


Due to time constraints the method for driving them at variable speeds was not implemented. The steps to implement it
would involve either a separate function analyzing the amplitude of the inputs, or to compare the outputs of the RNN,
with higher certainty leading to higher speed.


The latency saw definite improvement from the starting point of the previous work, showing an improvement by observation of the video made by the last group. During testing of the exoskeleton made by the last group,
the exoskeleton finished a full movement from start to stop approximately one second after the user finished the movement. During testing of the RNN code in Labview a full iteration took 230 ms.


\subsection{Future Work}

Any future projects should consider as main priorities acquiring and setting up a motor controller to enable feedback on any other aspect of the future work
as well as solving the issues with the \acrshort{rnn} training in Python.

Once the arm moves, two main directions could be considered. The first mounting it on a person, 