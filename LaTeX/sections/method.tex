\section{Method}
\label{section:method}
%In this section, the method used to find an answer to the research questions should be presented. 

%If this report presents results from a literature search, this means providing sufficient information 
%for allowing someone else to repeat the literature search and compare the results. I.e., a search using 
%the phrases a, b, and c, was made in database x, y and z on the date Month Date, Year (e.g., July 31st, 
%2021). The search resulted in x hits. Then, information on how you chose which works to include in this 
%report should be provided. The references should be used for answering your research questions.

%If the work reports on an experiment, this part should provide information about the experimental setup, 
%how the experiment was conducted, how data was collected and analyzed etc. Motivate methodological choices 
%through references. Also an experiment should be presented with sufficient detail such that it can be 
%repeated by someone else.

\subsection{Signal aqusition}
With a setup consisting of a Digilent Analog Discovery studio measurement device, two Myoware muscle sensor kits, various weights from two kilograms to ten kilograms and a resistance band.
Measurement data is acquired using the Digilent Waveforms program. In Waveforms a script was created with javascript code where data is labeled automatically from 0-3 with each state representing a different position,
0=resting, 1= up, 2= static hold, 3=down. With the use of a red and green Led lights, the user gathering the data would move depending on which light was glowing 


\subsection{RNN lstm}

\subsection{labview}

What we have done so far: 

\begin{itemize}
    
    \item Made a new fixture for the arm to allow easier access to it in exchange for disconnecting the shoulder motor for now.
    
    \item Used the Myoware 1.0 to sample a signal and tried to do the same with Myoware 2.0.
    
    \item Installed the needed software for the roboRIO to work and started investigating the use of LabView in tandem with it.
    
\end{itemize}
What we still need to do:
\begin{itemize}    
    \item Set up a shared database and data sampling method.
    
    \item Start sampling
    
    \item Choose AI method
    
    \item Either find a prebuilt or build an AI
\end{itemize}