\section{Method}
\label{section:method}
%In this section, the method used to find an answer to the research questions should be presented. 

%If this report presents results from a literature search, this means providing sufficient information 
%for allowing someone else to repeat the literature search and compare the results. I.e., a search using 
%the phrases a, b, and c, was made in database x, y and z on the date Month Date, Year (e.g., July 31st, 
%2021). The search resulted in x hits. Then, information on how you chose which works to include in this 
%report should be provided. The references should be used for answering your research questions.

%If the work reports on an experiment, this part should provide information about the experimental setup, 
%how the experiment was conducted, how data was collected and analyzed etc. Motivate methodological choices 
%through references. Also an experiment should be presented with sufficient detail such that it can be 
%repeated by someone else.

\subsection{Signal aqusition}
A setup consisting of a Digilent Analog Discovery studio measurement device, two Myoware muscle sensor kits, various weights from two kilograms to ten kilograms and a resistance band was used during the signal acquisition.
When using the Myoware sensors it is connected to the Analog Discovery oscilloscopes, ground and a 3.3 voltage input. Two oscilloscopes are connected to the two Myoware sensors one for the bicep and one for the tricep.
Measurement data is acquired using the Digilent Waveforms program. In Waveforms a script was created with javascript code where data is labeled from 0-3 with each state representing a different action,
0=resting, 1= up, 2= static hold, 3=down. With the use of a red and green LED light, the user connected to the Myoware and gathered the data starts the script and then picks the following options, 
which arm is used, what weight and if the weight going up or down to start with.      


\subsection{\acrfull{rnn}}
With the objective of having a reliable method for \acrshort{emg} based continuous motion prediction we chose to use a \acrshort{rnn} as our AI model.
Since the purpose of the project is to end with a working exoskeleton based on the RoboRIO system, an embedded solution had to be found. This resulted
in two different \acrshort{rnn} implementations for the same model:
\begin{itemize}

    \item \textbf{Python implementation:} The first implementation was done using Python and the PyTorch library. This implementation was used to train the model, test its accuracy and extract its parameters.
    
    \item \textbf{LabView implementation:} The second implementation was done using LabView and the RoboRIO system. This implementation was used to import the python parameters to use the \acrshort{rnn} in a real-time scenario.

\end{itemize}

Before going through with implementing the model, a design was made. The final model is based on the work of \cite{RNNEMG} as their \acrfull{lstm} implementation provided promising results both in accuracy and
inference time. The model consists of three layers:
\begin{itemize}

    \item Input layer

    \item First \acrshort{lstm} layer

    \item Second \acrshort{lstm} layer
    
    \item Output layer

\end{itemize}

\cite{RNNEMG} proposed different ranges of values for parameters such as the number of nodes in each of the \acrshort{lstm} layers, the learning rate and the number of epochs in the training process
which were then optimized by a \acrfull{gwo} algorithm for their specific dataset. However, due to hardware restrictions, running such optimizer was not possible. Instead, the parameters were
chosen by trial and error, process during which we realized that the model provided a perfect accuracy for our dataset leading to the conclussion that the number of epochs could be reduced to 1 and
the final model would be chosen by the one that provided the lowest inference time. Figure \ref{fig:rnn_struct} shows the structure of our final model after the selection process and table \ref{table:rnn_params} shows the parameters used.

\begin{figure}
    \centering
    \includegraphics[scale=0.25]{images/rnn_struct.png}
    \caption{Structure of the final \acrshort{rnn} model}
    \label{fig:rnn_struct}
\end{figure}

\begin{table}[]
    \centering
    \begin{tabular}{|c|c|}
    \hline
    \textbf{Parameter} & \textbf{Value} \\ \hline
    Number of nodes in the first \acrshort{lstm} layer & 15 \\ \hline
    Number of nodes in the second \acrshort{lstm} layer & 8 \\ \hline
    Learning rate & 0.00502 \\ \hline
    Number of epochs & 1 \\ \hline
    \end{tabular}
    \caption{Parameters of the \acrshort{rnn} model}
    \label{table:rnn_params}
\end{table}

\subsection{labview}

What we have done so far: 

\begin{itemize}
    
    \item Made a new fixture for the arm to allow easier access to it in exchange for disconnecting the shoulder motor for now.
    
    \item Used the Myoware 1.0 to sample a signal and tried to do the same with Myoware 2.0.
    
    \item Installed the needed software for the roboRIO to work and started investigating the use of LabView in tandem with it.
    
\end{itemize}
What we still need to do:
\begin{itemize}    
    \item Set up a shared database and data sampling method.
    
    \item Start sampling
    
    \item Choose AI method
    
    \item Either find a prebuilt or build an AI
\end{itemize}